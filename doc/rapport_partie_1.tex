\documentclass{article}

\usepackage[utf8]{inputenc}
\usepackage[top=2cm, left=2cm, right=2cm, bottom=2cm]{geometry}

\title{Projet onto}
\author{Joris Argentin, Xavier Bussell}
\date{M22 - 22 mars - 2019/2020}

\begin{document}

\maketitle

\section{Constitution des corpus}

Nous avons choisis de générer notre corpus à partir des articles contenant le terme MeSH : \emph{Microsatellite instability}.
Nous avons obtenus un corpus composés des résumés de $3093$ articles.
Ce corpus est au format XML.

\section{Fouille de données}

\subsection{Pré-traitement des données}

Pour la fouille de données, nous nous intéressons aux termes MeSH et aux qualifiers.
Le pré-traitement que nous avons appliqué se déroule en 2 étapes :
\begin{enumerate}
    \item Extraction des termes MeSH et qualifiers.
    \item formatage pour l'outil de fouille (Weka).
\end{enumerate}

\subsubsection{Extraction des termes et qualifiers}

Nous avons exécutés une requête XQUERY afin d'extraire les balises correspondantes.
Si il n'y a pas de qualifier associé au terme MeSH, un \emph{NA} sera inséré.
Le résultat de la requête fourni un fichier XML contenant pour chaque article les termes et qualifiers associés.

\end{document}